\documentclass[a4paper,10pt]{article}
\usepackage[polish]{babel}
\usepackage[utf8]{inputenc}
\usepackage{polski}
\usepackage[T1]{fontenc}
\usepackage{enumerate}
\usepackage{indentfirst}
\usepackage{graphicx}
\author{Mateusz Gałażyn}
\title{Symulacja fluktuacji magnetyzacji idealnego paramagnetyka w zewnętrznym polu magnetycznym}
\setlength{\topmargin}{0mm}
\frenchspacing
\begin{document}
	\maketitle
	\section{Cel projektu}
	Celem projektu jest symulacja fluktuacji idealnego paramagnetyka w zewnętrznym polu magnetycznym opisanego za pomocą modelu Isinga. W tym celu została napisana aplikacja w Javie ilustrująca rozkład spinów w materiale magnetycznym oraz skrypt w Octave, który generuje wykresy zależności fluktuacji (względnej i bezwzględnej) magnetyzacji, a także samej magnetyzacji dla próbek o wielkości $N=200$ oraz $N=2000$ spinów.
	\section{Opis teoretyczny}
	Energia jednowymiarowego łańcucha spinów w modelu Isinga jest określona zależnością \cite{az}:
	\[ E = -\sum^{N-1}_{i=1} J s_{i} s_{i+1} - H M \]
	Magnetyzacja materiału wynosi:
	\begin{equation}
	\label{magnetyzacja}
	 M = \gamma \sum^{N}_{i=1}s_{i} = \gamma N \langle s \rangle 
	\end{equation}
	gdzie $J$ - oddziaływanie między węzłami, $s_{i}$ - spin i-tej cząstki w łańcuchu $s_{i} \in \{-1,1\}$, $\gamma$ - współczynnik proporcjonalności oddziaływania jednego elementu łańcucha z polem magnetycznym $H$, $N$ - ilość cząstek w łańcuchu, $\langle s \rangle$ - wartość średnia spinu w łańcuchu. W przypadku idealnego paramagnetyka mamy do czynienia z brakiem oddziaływania pomiędzy poszczególnymi spinami, więc współczynnik $J = 0$. Dla paramagnetyka energia oddziaływania pojedynczej cząstki wynosi $E_i = - \gamma H s_{i}$, a dla substancji wyraża się wzorem:
	\[ E = \sum^{N}_{i=1} E_{i} =-\gamma H \sum^{N}_{i=1}s_{i} = -\gamma H N \langle s \rangle\]
	Rozkład prawdopodobieństwa wystąpienia danego spinu dla cząstki jest rozkładem kanonicznym. Jednocząstkowa suma statystyczna wynosi\cite{az}:
	\[ z_{i} = e^{- \beta \gamma H } + e^{\beta \gamma H} = 2 \cosh(\beta \gamma H); \hspace{1cm} \beta = \frac{1}{k_{B}T}  \]
	Prawdopodobieństwo wystąpienia spinu w łańcuchu:
	\begin{equation}
	\label{alfa}
	p(s_i) = \left\{ \begin{array}{lll}
	\alpha & = e^{\beta \gamma H} / z_{i}, & \mbox{ gdy } s_i = +1 \\
	1-\alpha & = e^{ -\beta \gamma H} / z_{i}, & \mbox{ gdy } s_i = -1
	\end{array} \right.
	\end{equation}
	Wartość średnia spinu:
	\[ \langle s \rangle = p(s_i=1) - p(s_i=-1) = 2 \alpha -1 = \tanh (\beta \gamma H) \]
	Fluktuacja magnetyzacji jest wyrażona następującą zależnością\cite{af}:
	\begin{equation}
	\label{fluktuacja}
	\sigma_{M}^{2} = \langle (M-\langle M \rangle)^{2}\rangle \\ = \gamma^{2} \sum^{N}_{i=1} \langle \left( 1-\langle s \rangle \right)^{2} \rangle  = \gamma^{2} N  \left( 1 - \langle s \rangle^{2} \right) \\ = \gamma^{2} N (1-\tanh^{2} (\beta \gamma H))
	\end{equation}
	\section{Wyniki symulacji}
	Za pomocą skryptu {\bf simulation.m } zostały wygenerowane wykresy fluktuacji oraz magnetyzacji dla paramagnetyka o ilości spinów $N=200$ oraz $N=2000$. Dla uproszczenia sytuacji zostały przyjęte stałe $k_{B}=\gamma=1$.
	

		\includegraphics[width=1\textwidth]{fl_200}

		\includegraphics[width=1\textwidth]{fl_2000.png}
		
		\includegraphics[width=1\textwidth]{rel_fl_200.png}

		\includegraphics[width=1\textwidth]{rel_fl_2000.png}
	Na wykresach fluktuacji jest zauważalne tworzenie się porządku w sieci spinów (spadek wartości $\sigma_{M}^{2}$ oraz $\sigma_{M}/|M|$) wraz z wzrastającą wartością $|\beta H|$.\\
		\includegraphics[width=1\textwidth]{M_200.png}

		\includegraphics[width=1\textwidth]{M_2000.png}
	Zauważalne jest asymptotyczne zbieganie magnetyzacji do określonych wartości dla dużych natężeń pola magnetycznego - następuje wtedy uporządkowanie większości spinów w jednym kierunku. Dla małych natężeń pola magnetycznego wykres magnetyzacji jest prawie liniowy - na tym obszarze jest dobrze spełnione prawo Curie.\\
	Na wszystkich wykresach jest zauważalne zbieganie się wyników z symulacji z krzywymi teoretycznymi wraz z wzrostem liczby spinów.
	\section{Implementacja}
	Dołączona aplikacja napisana w Javie symuluje rozkład spinów w idealnym paramagnetyku. Białe kwadraty reprezentują cząstki o spinie równym  $s_{i}\nolinebreak=\nolinebreak1$, a czarne odpowiednio $s_{i}\nolinebreak=\nolinebreak-1$. Dodatnia wartość pola oraz spinu odpowiada zwrotowi wektora wychodzącego przed płaczszczyznę monitora. Dla uproszczenia badanego przypadku zostały przyjęte wartości stałych równe jedności: $k_{B}=\gamma=1$. Wartości spinów są generowane za pomocą generatora liczb pseudolosowych z zakresu $[0,1]$, a następnie wylosowana wartość jest porównywana z parametrem $\alpha$ (\ref{alfa}) co pozwala przypisać poszczególnym komórkom na wykresie odpowiadające im wartości spinów. Magnetyzacja i fluktuacja jest obliczana odpowiednio ze wzorów: (\ref{magnetyzacja}) i (\ref{fluktuacja}). \\ \\
	Zawartość załączonych plików źródłowych:\\
		\begin{description}
		\setlength{\itemindent}{0cm}
		\item[Graph.java] - Klasa rysujaca wykres.
		\item[GraphUpdater.java] - Klasa odpowiedzialna za wyznaczenie rozkładu spinów.
		\item[ModificationListener.java] - Klasa odpowiedzialna za przechwytywanie zdarzeń z interfejsu.
		\item[Window.java] - Klasa tworząca interfejs.
		\item[simulation.m] - Skrypt w Octave symulujący zachowanie się paramagnetyka i generujący wykresy fluktuacji.
		\end{description}
		
	\begin{thebibliography}{99}
		\bibitem{az} A. Zagórski:
			\emph{Fizyka Statystyczna},
			 OFPW, Warszawa 1994
		\bibitem{af} A. Fronczak:
			\emph{Zadania i problemy z rozwiązaniami z termodynamiki i fizyki statystycznej},
			 OFPW, Warszawa 2006
		\end{thebibliography}
	\end{document}
